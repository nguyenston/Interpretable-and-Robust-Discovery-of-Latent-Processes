\newcommand{\methodname}{\textsf{ACDC}\xspace}

%%%%%%%%%%%%%%%%%%
%%% Commenting %%% 
%%%%%%%%%%%%%%%%%%

%% NA: needs attention (rough writing whose correctness needs to be verified)
%% TBD: instructions for how to fix a gap ("Describe the propagation by ...")
%% PROBLEM: bug or missing crucial bit 

%% use \fXXX versions of these macros to put additional explanation into a footnote (in the margin).  
%% The idea is that we don't want to interrupt the flow of the paper or make it 
%% impossible to read because there are a bunch of comments.

%% NA's (and TBDs, those less crucially) should be written so 
%% that they flow with the text.

\definecolor{WowColor}{rgb}{.75,0,.75}
\definecolor{SubtleColor}{rgb}{0,0,.50}

% inline
\newcommand{\NA}[1]{\textcolor{SubtleColor}{ {\tiny \bf ($\star$)} #1}}
\newcommand{\LATER}[1]{\textcolor{SubtleColor}{ {\tiny \bf ($\dagger$)} #1}}
\newcommand{\TBD}[1]{\textcolor{SubtleColor}{ {\tiny \bf (!)} #1}}
\newcommand{\PROBLEM}[1]{\textcolor{WowColor}{ {\bf (!!)} {\bf #1}}}

% as margin notes
\newcounter{margincounter}
\newcommand{\displaycounter}{{\arabic{margincounter}}}
\newcommand{\incdisplaycounter}{{\stepcounter{margincounter}\arabic{margincounter}}}

\newcommand{\fTBD}[1]{\textcolor{SubtleColor}{$\,^{(\incdisplaycounter)}$}\marginpar{\tiny\textcolor{SubtleColor}{ {\tiny $(\displaycounter)$} #1}}}

\newcommand{\fPROBLEM}[1]{\textcolor{WowColor}{$\,^{((\incdisplaycounter))}$}\marginpar{\tiny\textcolor{WowColor}{ {\bf $\mathbf{((\displaycounter))}$} {\bf #1}}}}

\newcommand{\fLATER}[1]{\textcolor{SubtleColor}{$\,^{(\incdisplaycounter\dagger)}$}\marginpar{\tiny\textcolor{SubtleColor}{ {\tiny $(\displaycounter\dagger)$} #1}}}

\DeclareRobustCommand{\suppresscomments}{
	% For submission, make all render blank.
	\renewcommand{\LATER}[1]{}
	\renewcommand{\fLATER}[1]{}
	\renewcommand{\TBD}[1]{}
	\renewcommand{\fTBD}[1]{}
	\renewcommand{\PROBLEM}[1]{}
	\renewcommand{\fPROBLEM}[1]{}
	\renewcommand{\NA}[1]{##1}  %% Note, NA's pass through!
}


\newcommand{\charalphabetmacro}[3]{
	% #1: annotation prefix
	% #2: annotation command (must be a single command!)
	% #3: characters to define annotations for
	% Create a macros of the form
	%   \newcommand{\#1##1}{#2{##1}} for each ##1 in #3
	\def\mydeffoo##1{\expandafter\def\csname #1##1\endcsname{#2{##1}}}
	\def\mydefallfoo##1{\ifx##1\mydefallfoo\else\mydeffoo##1\expandafter\mydefallfoo\fi}
	\expandafter \mydefallfoo #3\mydefallfoo
}

\newcommand{\texalphabetmacro}[3]{
	% #1: annotation prefix
	% #2: annotation command (must be a single command!)
	% #3: latex symbol command text to define annotations for
	% Create a macros of the form
	%   \newcommand{\#1##1}{#2{\##1}} for each ##1 in #3
	\def\mydeftex##1{\expandafter\def\csname #1##1\endcsname{#2{\csname ##1\endcsname}}}
	\def\mydefalltex##1{\ifx\mydefalltex##1\else\mydeftex{##1}%
			\lowercase{\mydeftex{##1}}\expandafter\mydefalltex\fi}
	\expandafter \mydefalltex #3\mydefalltex
}

\newcommand{\upperCaseRomanLetters}{ABCDEFGHIJKLMNOPQRSTUVWXYZ}
\newcommand{\lowerCaseRomanLetters}{abcdefghijklmnopqrstuvwxyz}
\newcommand{\lowerCaseRomanLettersNoMHT}{abcdefgijklnopqrsuvwxyz}
\newcommand{\lowerCaseRomanLettersNoG}{abcdefhijklmnopqrstuvwxyz}
\newcommand{\lowerCaseRomanLettersNoMF}{abcdeghijklnopqrstuvwxyz}


\newcommand{\lowerCaseGreekLetters}{{alpha}{beta}{gamma}{delta}{epsilon}{zeta}{eta}{theta}{iota}{kappa}{lambda}{mu}{nu}{xi}{omicron}{pi}{rho}{sigma}{tau}{upsilon}{phi}{chi}{psi}{omega}}
\newcommand{\lowerCaseGreekLettersNoEta}{{alpha}{beta}{gamma}{delta}{epsilon}{zeta}{theta}{iota}{kappa}{lambda}{mu}{nu}{xi}{omicron}{pi}{rho}{sigma}{tau}{upsilon}{phi}{chi}{psi}{omega}}
\newcommand{\upperCaseGreekLettersInLaTeX}{{Gamma}{Delta}{Theta}{Lambda}{Xi}{Pi}{Sigma}{Upsilon}{Phi}{Psi}{Omega}}


% Bar everything shortcuts
\charalphabetmacro{bar}{\bar}{\upperCaseRomanLetters}
\charalphabetmacro{bar}{\bar}{\lowerCaseRomanLetters}

\charalphabetmacro{b}{\boldorbar}{\upperCaseRomanLetters}
\charalphabetmacro{b}{\boldorbar}{\lowerCaseRomanLettersNoMF} % avoid clash with \bm and \bf
\newcommand{\boldf}{\bm{f}} % only need \boldf, since \barf already defined above

\texalphabetmacro{b}{\boldorbar}{\lowerCaseGreekLettersNoEta} % avoid \beta <-> \bar{\eta} clash
\newcommand{\bareta}{\bar{\eta}}
\newcommand{\boldeta}{\bm{\eta}}
\texalphabetmacro{b}{\boldorbar}{\upperCaseGreekLettersInLaTeX}

% Mathcal shortcuts
\charalphabetmacro{mc}{\mathcal}{\upperCaseRomanLetters}

\newcommand{\widehatmathcal}[1]{\widehat{\mathcal{#1}}}
\charalphabetmacro{hmc}{\widehatmathcal}{\upperCaseRomanLetters}

% Hatted shortcuts
\charalphabetmacro{h}{\widehat}{\upperCaseRomanLetters}
\charalphabetmacro{h}{\widehat}{\lowerCaseRomanLettersNoMHT}
\texalphabetmacro{h}{\widehat}{\lowerCaseGreekLetters}
\texalphabetmacro{h}{\widehat}{\upperCaseGreekLettersInLaTeX}

% Bold hatted shortcuts
\newcommand{\boldhat}[1]{\mathbf{\widehat{#1}}}
\charalphabetmacro{bh}{\boldhat}{\upperCaseRomanLetters}
\charalphabetmacro{bh}{\boldhat}{\lowerCaseRomanLetters}
\texalphabetmacro{bh}{\boldhat}{\lowerCaseGreekLetters}
\texalphabetmacro{bh}{\boldhat}{\upperCaseGreekLettersInLaTeX}


%Wide hatted shortcuts
\charalphabetmacro{wh}{\widehat}{\upperCaseRomanLetters}
\charalphabetmacro{wh}{\widehat}{\lowerCaseRomanLettersNoMHT}

% Tilde shortcuts 
\charalphabetmacro{td}{\tilde}{\upperCaseRomanLetters}
\charalphabetmacro{td}{\tilde}{\lowerCaseRomanLetters}
\texalphabetmacro{td}{\tilde}{\lowerCaseGreekLetters}
\texalphabetmacro{td}{\tilde}{\upperCaseGreekLettersInLaTeX}


\newcommand{\defas}{:=}

% Stacked Relations
%%%%%%%%%%%%%%%%%%%
\newcommand{\stk}[2]{\ensuremath{\stackrel{\text{#2}}{#1}}}
\newcommand{\stkm}[2]{\ensuremath{\stackrel{#2}{#1}}}


\newcommand{\abscont}{\ll}  % absolutely continuous 
\newcommand{\nnreals}{\reals_{+}}
\newcommand{\cconj}[1]{\overline{#1}}  % complex conjugation 
\newcommand{\im}{\operatorname{im}}

\newcommand{\fracexp}[3]{\left(\frac{#1}{#2}\right)^{#3}}
\newcommand{\gammafrac}[2]{\frac{\Gamma(#1)}{\Gamma(#2)}}
\newcommand{\textder}[2]{\dee #1/\dee #2}

\newcommand{\distSS}{\distNamed{SS}} % symmetric stable distribution 
\newcommand{\ii}{i} % imaginary unit 
\newcommand{\erfc}{\operatorname{erfc}}
\newcommand{\logit}{\psi}
\newcommand{\logistic}{\phi}
\newcommand{\derivop}[1]{\grad^{#1}}
\newcommand{\hessian}{\derivop{2}}
\newcommand{\mean}[1]{\bar#1}
\newcommand{\indicatorfn}{\mathds{1}}
\newcommand{\id}{\operatorname{id}}

\newtheorem*{notation}{Notation}
\newcommand{\littleo}{o}
\newcommand{\littleoP}{\littleo_{P}}
\newcommand{\littleoPouter}{\littleo_{P^{+}}}
\newcommand{\bigo}{O}
\newcommand{\bigoP}{\bigo_{P}}
\newcommand{\bigoPouter}{\bigo_{P^{+}}}
\newcommand{\convto}{\overset{\Pr_{\trueparam}}{\rightarrow}}
\newcommand{\convtodist}{\overset{d}{\rightarrow}}
\newcommand{\empvar}[1]{\widehat\sigma^{2}(#1)}
% \newcommand{\empcov}[1]{\widehat\Sigma(#1)}
\newcommand{\empcov}{\widehat\Sigma}
\newcommand{\empmean}[1]{\widehat\mu(#1)}

\newcommand{\statistic}[1]{T_{#1}}
\newcommand{\p}{p}


%--------------------------------------------------------------------------------------
% Sets
%--------------------------------------------------------------------------------------
\newcommand{\intersection}{\cap}

%--------------------------------------------------------------------------------------
% Bayesian stuff
%--------------------------------------------------------------------------------------


\newcommand{\model}[1]{\mcM^{(#1)}}
\newcommand{\paramSpace}[1]{\Theta^{(#1)}}

% Data
\newcommand{\truesym}{\circ}
\newcommand{\dataspace}{\mcX}
\newcommand{\obsspace}{\mathbb X}


\newcommand{\datarv}{\mathbf{X}}
\newcommand{\dataarg}[1]{x_{1:#1}}
\newcommand{\datarvarg}[1]{X_{1:#1}}
\newcommand{\datapart}[1]{\dataarg{(#1)}}
\newcommand{\alldata}{\dataarg{\infty}}
\newcommand{\alldatarv}{\datarvarg{\infty}}
\newcommand{\datacopy}{Y}
\newcommand{\datarvcopy}{\mathbf Y}
\newcommand{\datatrue}{\data_{\truesym}}
\newcommand{\obs}[1]{x_{#1}}
\newcommand{\obsrv}[1]{X_{#1}}
\newcommand{\datameanarg}[1]{\bar\data_{#1}}
\newcommand{\datarvmeanarg}[1]{\bar\datarv_{#1}}
% \newcommand{\datameanarg}[1]{\empmean{\dataarg{#1}}}


% Posterior and priors
\newcommand{\basedist}{\lambda}
\newcommand{\postdist}[1]{\Pi_{#1}}
\newcommand{\postdistfull}[2]{\Pi(#1 \given #2)}
\newcommand{\ppostdist}[2]{\Pi_{#1}^{#2}}
\newcommand{\postdensity}[1]{\pi_{#1}}
\newcommand{\postdensityfull}[2]{\pi(#1 \given #2)}
\newcommand{\ppostdensity}[2]{\pi_{#1}^{#2}}
\newcommand{\priordist}{\postdist{0}}
\newcommand{\priordensity}{\postdensity{0}}



%\newcommand{\lik}[2]{f(#1 \given #2)}
%\newcommand{\loglik}[2]{\mcL(#1 \given #2)}
%\newcommand{\Eloglik}[1]{\bar\mscL(#1)}
% log-likelihood
\newcommand{\likdist}[1]{P_{#1}}
\newcommand{\likfun}[1]{p_{#1}}
\newcommand{\lik}[2]{\likfun{#2}(#1)}
\newcommand{\loglik}[2]{\ell_{#2}(#1)}
\newcommand{\loglikfun}[1]{\ell_{#1}}
\newcommand{\Eloglik}[1]{\bar\ell_{#1}}
\newcommand{\optEloglik}{\bar\ell_{\optsym}}
\newcommand{\Eloglikdiff}[1]{\Delta_{#1}}
\newcommand{\gradloglik}[2]{\dot\ell_{#2}(#1)}
\newcommand{\gradloglikfun}[1]{\dot\ell_{#1}}
\newcommand{\hessloglik}[2]{\ddot\ell_{#2}(#1)}
\newcommand{\hessloglikfun}[1]{\ddot\ell_{#1}}

% data distributions 
\newcommand{\priormarginal}{m_\textrm{prior}}
\newcommand{\postmarginal}{m_\textrm{post}}
%\newcommand{\marginallik}[1]{p_{#1}}
%\newcommand{\pmarginallik}[2]{p(#1)^{#2}}

% parameter
\newcommand{\opt}{\star}
\newcommand{\optsym}{\star}

%
%\newcommand{\paramsample}{\vartheta}
%\newcommand{\bbparam}{\param^{\bbsym}}
%\newcommand{\bbparamsample}{\paramsample^{\bbsym}}
%\newcommand{\bbparamsamplecopy}{\paramsample^{\bbsym\prime}}
%\newcommand{\bbparamsamplenum}{\paramsample^{\bbsym}_{\numobs}}
%\newcommand{\optparams}{\paramspace_{o}}
%\newcommand{\mle}[1]{\widehat\param}
%
%% Model selection
%\newcommand{\modelspace}{\mathfrak{M}}
%\newcommand{\model}{\mathfrak{m}}
%\newcommand{\optmodel}{\model_{\optsym}}
%\newcommand{\modelmarginallik}[2]{\marginallikfull{#1 \given #2}}
%\newcommand{\modelpriordist}{Q_{0}}
%% \newcommand{\modelpostdist}[1]{Q_{#1}}
%\newcommand{\modelpostdistfull}[2]{Q(#1 \given #2)}
%\newcommand{\priorlogodds}{\rho}
%\newcommand{\lmldiff}[1]{\Lambda_{#1}}
%\newcommand{\optmodelparam}[1]{\param_{#1 \optsym}}
%
%% Fisher information
%\newcommand{\Ehessloglik}[1]{J_{#1}}
%\newcommand{\Vargradloglik}[1]{I_{#1}}
%
%\newcommand{\paramcovmat}{C}
%\newcommand{\paramcov}[2]{\paramcovmat_{#1#2}}
%
%
%\newcommand{\credset}[2]{C_{#1}(#2)}
%\newcommand{\credsetintersect}[3]{I_{#1}(#2, #3)}
%\newcommand{\normalcredsetconst}[1]{c_{#1}}
%
%% data distributions 
\newcommand{\datadist}{P_{o}}
\newcommand{\distDisc}{\mcD_{\mathrm{dist}}}
\newcommand{\compDisc}{\mcD_{\mathrm{comp}}}
\newcommand{\compDiscEst}{\widehat{\mcD}_{\mathrm{comp}}}
%\newcommand{\obsdist}{P_{\truesym}}
%\newcommand{\empdist}{\Pr_{\numobs}}
%\newcommand{\datasum}{Z}
%\newcommand{\datadiff}{W}
%
%
%% BayesBag
%\newcommand{\bbsym}{*} % BayesBag symbol
%\newcommand{\bsnumobs}{M}
%\newcommand{\bbEE}{\EE^{\bbsym}} % expectation wrt to BayesBag
%\newcommand{\bsindrvsym}{L}
%\newcommand{\bsindsym}{\ell}
%\newcommand{\bsind}[1]{\bsindsym_{#1}}
%\newcommand{\bsindrv}[1]{\bsindrvsym_{#1}}
%\newcommand{\bsindrvs}[1]{\bsindrvsym_{1:#1}}
%\newcommand{\bscount}[1]{K_{#1}}
%\newcommand{\bscounts}{K_{1:\numobs}}
%\newcommand{\bsobs}[1]{\obs{#1}^{\bbsym}}
%\newcommand{\bsobsrv}[1]{\obsrv{#1}^{\bbsym}}
%\newcommand{\bsdata}{\data^{\bbsym}}
%\newcommand{\bsdatarv}{\datarv^{\bbsym}}
%\newcommand{\bsdatasample}[1]{\data^{\bbsym}_{(#1)}}
%\newcommand{\bsdataarg}[1]{\bsdata_{1:#1}}
%\newcommand{\bsdatarvarg}[1]{\bsdatarv_{1:#1}}
%\newcommand{\bsdatacopy}[1]{\datacopy^{\bbsym}_{1:#1}}
%\newcommand{\bbpostdist}[1]{\postdist{#1}^{\bbsym}}
%\newcommand{\bbpostdistfull}[2]{\Pi^{\bbsym}(#1 \given #2)}
%\newcommand{\bbpostdensity}[1]{\postdensity{#1}^{\bbsym}}
%\newcommand{\bbpostdensityfull}[2]{\postdensity{}^{\bbsym}(#1 \given #2)}
%\newcommand{\bsdatameanarg}[1]{\bar\data^{\bbsym}_{#1}}
%\newcommand{\bsdatarvmeanarg}[1]{\bar\datarv^{\bbsym}_{#1}}
%% \newcommand{\bsdatameanarg}[1]{\empmean{\bsdataarg{#1}}}
%% \newcommand{\bsdatarvmeanarg}[1]{\empmean{\bsdatarvarg{#1}}}
%%\newcommand{\bsdatameanargrv}[1]{\bar\data_{\bsindrvsym}}
%\newcommand{\bbempdist}{\Pr_{\numobs}^{\bbsym}}
%% \newcommand{\bbmodelpostdist}[1]{Q_{#1}^{\bbsym}}
%\newcommand{\bbmodelpostdistfull}[2]{Q^{\bbsym}(#1 \given #2)}
%%\newcommand{\indexvar}[1]{\bsindrvsym_{#1}}
%
%
%\newcommand{\bsscale}{c}
%%\newcommand{\optscale}{\bsscale_{\text{opt}}}
%%\newcommand{\optscaleest}{\widehat\bsscale_{\text{opt}}}
%\newcommand{\optMasympt}{\bsnumobs_{\infty,\text{opt}}}
%\newcommand{\optMasymptest}{\widehat\bsnumobs_{\infty,\text{opt}}}
%\newcommand{\optMfs}{\bsnumobs_{\text{fs,opt}}}
%\newcommand{\optMfsest}{\widehat\bsnumobs_{\text{fs,opt}}}
%\newcommand{\optMest}{\widehat\bsnumobs_{\text{opt}}}
%\newcommand{\modelmismatch}{\mathcal{I}}
%\newcommand{\nan}{\textsf{NA}}
%\newcommand{\concconst}{C_{\numobs}}
%
%% Split Predictive Checks
%
%
%\newcommand{\pcobsrv}{X_{\text{obs}}}    % X
%\newcommand{\pcobsobs}[1]{\obsrv{#1}}
%\newcommand{\pcnewrv}{X_{\text{ho}}}  % X^rep
%\newcommand{\pcreprv}{X_{\text{pred}}}  % X^rep
%\newcommand{\pcrepobs}[1]{\obsrv{#1}_{\text{pred}}}
%% single SPC
%\newcommand{\spc}{\text{SPC}}  % SPC index
%\newcommand{\spcpvalue}[1]{ \p_{\spc}(#1)}
%\newcommand{\spcprop}{q}
%\newcommand{\spcobsrv}{\datarv_{\text{obs}}}    % X^obs
%\newcommand{\spcobsobs}[1]{X_{\text{obs},#1}}
%\newcommand{\spcnewrv}{\datarv_{\text{ho}}}    % X^new
%\newcommand{\spcnewobs}[1]{X_{\text{ho}, #1}}
%\newcommand{\spcreprv}{\datarv_{\text{pred}}}  % X^rep
%\newcommand{\spcrepobs}[1]{X_{\text{pred},#1}}
%\newcommand{\spcobssize}{\lceil \spcprop\numobs \rceil}
%\newcommand{\spcnewsize}{ \numobs - \spcNobs}
%\newcommand{\spcNobs}{\numobs_{\text{obs}}}
%\newcommand{\spcNnew}{\numobs_{\text{ho}}}
%\newcommand{\spcpostmarginal}{\postmarginal^{\spc}}
%\newcommand{\intBVMapprox}{\widehat{\int}}
%% divided SPC
%\newcommand{\dspcK}{k}
%\newcommand{\dspc}{\text{dSPC}}
%\newcommand{\dspcdata}[1]{\datarv^{(#1)}}
%\newcommand{\dspcpval}[1]{\p_{\dspc}(#1)}
%\newcommand{\dspcobspval}[1]{u_{#1}}
%\newcommand{\dspcpdist}{\mathbb{P}_\dspcK}
%\newcommand{\dspcnumobs}{N_\dspcK}
%\newcommand{\dspcobs}{N_{\dspcK,\text{obs}}}
%\newcommand{\spcrepobsrv}{\mathbf{\datarv}^{\spc}_{\text{obs}}}
%\newcommand{\spcrepobsobs}[1]{\datarv^{\spc}_{\text{obs}}}
%\newcommand{\spcrepnewrv}{\mathbf{\datarv}_{\text{ho}}}
%
%
%\newcommand{\pvalbvmdist}[1]{ \p'_{\spc}(#1)}
%\newcommand{\pvalbvmdistball}[1]{ \p''_{\spc}(#1)}
%
%
%\newcommand{\pvaltoK}[1]{ \p'_{\spc}(#1)}
%\newcommand{\spcpvalueapprox}[1]{ \p''_{\spc}(#1)}
%\newcommand{\spcpvalball}[1]{ \p'''_{\spc}(#1)}
%\newcommand{\spcrepnewobs}[1]{\datarv_{\spc, #1}^{\text{ho}}}
%\newcommand{\spcKSpvalue}{ \p_{\text{KS}}(\pcobsrv)}
%\newcommand{\spcKSstat}{\sup_x|x - \KSempdist(x)|}
%\newcommand{\spcpvaluearg}[1]{\p{1:#1}}
%\newcommand{\spcpvalueobs}[1]{\p{#1}}
%\newcommand{\KSempdist}{\Pr_{\spcreptimes}}
%
%\newcommand{\asympmean}[1]{\nu_{#1}}
%\newcommand{\trueasympmean}{\nu_{\truesym}}
%\newcommand{\diffasympmean}[1]{\dot{\nu}_{#1}}
%\newcommand{\asympsd}{\sigma}
%\newcommand{\trueasympsd}{\asympsd_{\truesym}}
%\newcommand{\diffasympvar}[1]{\dot{\asympsd}(#1)}
%\newcommand{\ddiffasympvar}[1]{\ddot{\asympsd}(#1)}
%\newcommand{\mlevar}{\Sigma}
%\newcommand{\truemlevar}{\Sigma_{\optsym}}
%\newcommand{\distBVM}[2]{\widehat{\Pi}(#1\given#2)}
%\newcommand{\densityBVM}[2]{\widehat{\pi}(#1\given#2)}
%\newcommand{\paramstar}{\param'}
%\newcommand{\localparam}[1]{{\psi}_{#1}}
%\newcommand{\paramseq}[1]{\tilde{\param}_{#1}}
%\newcommand{\distKS}[1]{F_\mathrm{KS}(#1)}
%\newcommand{\qdistKS}[1]{K_{#1}}
%
%\newcommand{\bigoPindex}[1]{\bigo_{P_{#1}}}
%\newcommand{\standnormal}{ \mathcal{N}(0, 1)}
%\newcommand{\littleoPindex}[1]{\littleo_{P_{#1}}}
%
%\newcommand{\truesigma}{\sigma_{\truesym}}
%\newcommand{\liksd}{\sigma}
%\newcommand{\priorsd}{\tau}
%\newcommand{\priormean}{\mu}
%\newcommand{\sdratio}{\kappa}
%\newcommand{\ess}{\numobs_0}
%\newcommand{\unadjess}{r}
%\newcommand{\adjess}{r_{\text{adj}}}
%\newcommand{\samplemean}[1]{\overline{#1}}
%
%% dSPC
%\newcommand{\pvaldist}{\mathbb{P}_{k}}
%\newcommand{\pvalempdist}{\widehat{\mathbb{P}}_{k}}
%
%% empirical process notations
%\newcommand{\class}{\mathcal{C}}
%\newcommand{\fnclass}{\mathscr{F}}
%\newcommand{\VCclass}{\mathcal{C}}
%\newcommand{\card}{\text{Card}}
%\newcommand{\VCnum}[1]{\nu(#1)}
%\newcommand{\BB}[1]{\mathbb{G}^{(#1)}}
%\newcommand{\empGn}[2]{\mathbb{G}^P_{#1}#2}
%\newcommand{\empPn}[1]{\widehat{\mathbb{P}}_{#1}}
%\newcommand{\unifempGn}[2]{\tilde{\mathbb{G}}_{#1}#2}
%\newcommand{\indclass}{\mathscr{F}_{\text{ind}}}
%

%--------------------------------------------------------------------------------------
% Lp spaces
%--------------------------------------------------------------------------------------
\newcommand{\Lp}[1]{L^{#1}}
\newcommand{\Lpnorm}[2]{\staticnorm{#2}_{\Lp{#1}}}
\newcommand{\Lpinner}[3]{\inner{#2}{#3}_{\Lp{#1}}}
\newcommand{\Lparg}[2]{L^{#1}(#2)}
\newcommand{\Lpnormarg}[3]{\norm{#3}_{\Lparg{#1}{#2}}}
\newcommand{\Lpinnerarg}[4]{\inner{#3}{#4}_{\Lparg{#1}{#2}}}
\newcommand{\lpnorm}[2]{\staticnorm{#2}_{#1}}
\newcommand{\lpinner}[3]{\inner{#2}{#3}_{#1}}

%--------------------------------------------------------------------------------------
% Metrics
%--------------------------------------------------------------------------------------

\newcommand{\hellinger}[2]{\mathrm{H}(#1, #2)}
\newcommand{\sqhellinger}[2]{\mathrm{H}^{2}(#1, #2)}
\newcommand{\Esqhellinger}{\overline{\mathrm{H}^{2}}}
\newcommand{\postpredEsqhellinger}[1]{\overline{\mathrm{H}^{2}_{#1}}}
\newcommand{\bsEsqhellinger}{\overline{\mathrm{H}^{2}_{\bbsym}}}
\newcommand{\haffinity}[2]{\mathrm{A}(#1, #2)}
\newcommand{\Ehaffinity}{\overline{\mathrm{A}}}
\newcommand{\skl}[2]{\mathrm{SKL}(#1, #2)}
\newcommand{\Eskl}{\overline{\mathrm{SKL}}}
\newcommand{\dalpha}[3]{d_{#1}(#2 || #3)}


% ---------------------------------------------------------------------------------
% Topology
%---------------------------------------------------------------------------------------
\newcommand{\Int}[1]{\operatorname{Int}({#1})}

%--------------------------------------------------------------------------------------
% Vectors and matrices
%--------------------------------------------------------------------------------------
\newcommand{\boldone}{\mbf{1}} % Bold 1
\newcommand{\onevec}[1]{1_{#1}}
\newcommand{\ident}{\mbf{I}} % Identity matrix
\def\v#1{\mbi{#1}} % Vector notation
\def\norm#1{\left\|{#1}\right\|} % A norm with 1 argument
\newcommand{\onenorm}[1]{\norm{#1}_1} % L1 norm
\newcommand{\twonorm}[1]{\norm{#1}_2} % L2 norm
\newcommand{\infnorm}[1]{\norm{#1}_{\infty}} % Linfty norm
\newcommand{\opnorm}[1]{\norm{#1}_{op}} % Operator norm
\newcommand{\fronorm}[1]{\norm{#1}_{F}} % Frobenius norm
\newcommand{\nucnorm}[1]{\norm{#1}_{*}} % Nuclear norm
\def\staticnorm#1{\|{#1}\|} % A static norm that does not resize with input
\newcommand{\statictwonorm}[1]{\staticnorm{#1}_2} % L2 norm
\newcommand{\inner}[2]{\langle{#1},{#2}\rangle} % Inner product
\newcommand{\binner}[2]{\left\langle{#1},{#2}\right\rangle} % Inner product with pandable brackets
\def\what#1{\widehat{#1}}

\def\twovec#1#2{\left[\begin{array}{c}{#1} \\ {#2}\end{array}\right]}
\def\threevec#1#2#3{\left[\begin{array}{c}{#1} \\ {#2} \\ {#3} \end{array}\right]}
\def\nvec#1#2#3{\left[\begin{array}{c}{#1} \\ {#2} \\ \vdots \\ {#3}\end{array}\right]} % An n-vector with three arguments

%-------------------------------------------------------------------------------------------------------------------------------------------------------------------------------------------------------------------
% Words with special symbols
%-------------------------------------------------------------------------------------------------------------------------------------------------------------------------------------------------------------------
\def\Gronwall{Gr\"onwall\xspace}
\def\Renyi{R\'enyi\xspace}
\def\Holder{H\"older\xspace}
\def\Ito{It\^o\xspace}
\def\Nystrom{Nystr\"om\xspace}
\def\Schatten{Sch\"atten\xspace}
\def\Matern{Mat\'ern\xspace}
\def\Frechet{Fr\'echet\xspace}


\newcommand{\kl}[2]{\mathrm{KL}(#1 \mid #2)}
\newcommand{\kladaest}[2]{\mathrm{\tilde{KL}}(#1 \mid #2)}
\newcommand{\klest}[2]{\mathrm{\widehat{KL}}(#1 \mid #2)}
\newcommand{\klestsub}[4]{\mathrm{\widehat{KL}^{#1}}_{#2}(#3 \mid #4)}
\newcommand{\klestunif}[2]{\mathrm{\widehat{KL}}_{U}(#1 \mid #2)}



\newcommand{\data}[1]{x_{#1}}
\newcommand{\numobs}{N}
\newcommand{\numcomps}{K}
\newcommand{\numdims}{D}

\newcommand{\allparam}{\theta}
\newcommand{\param}{\phi}
\newcommand{\paramspace}{\Theta}


\newcommand{\Neff}{N_{\eff}}
\newcommand{\freq}{\nu}

\newcommand{\knn}{\tdk}


\newcommand{\discr}[2]{\mcD(#1\mid #2)}
\newcommand{\discrest}[2]{\widehat\mcD(#1\mid #2)}
\newcommand{\metric}{d}
\newcommand{\blmetric}{d_\mathrm{BL}}
\newcommand{\mmd}{\text{MMD}}


\renewcommand{\Pr}{\text{pr}}
\newcommand{\dee}{\mathrm{d}}
\newcommand{\op}{o_{P}}
\newcommand{\BLnorm}[1]{\Vert #1 \Vert_{\mathrm{BL}}}

\def\argmax{\operatornamewithlimits{arg\,max}}
\def\argmin{\operatornamewithlimits{arg\,min}}
\newcommand{\distNamed}[1]{{\sf{#1}}}
\newcommand{\distCat}{\distNamed{Categorical}}
\newcommand{\distNBinom}{\distNamed{NegBin}}
\newcommand{\distPoiss}{\distNamed{Poiss}}
\newcommand{\distUnif}{\distNamed{Unif}}
\newcommand{\distNorm}{\mathcal{N}}
\newcommand{\distSNorm}{\mathcal{SN}}
\newcommand{\distMulti}{\distNamed{Multi}}
\newcommand{\distGamma}{\distNamed{Gamma}}


\newcommand{\reals}{\ensuremath{\mathbb{R}}}
\newcommand{\ints}{\ensuremath{\mathbb{Z}}}
\newcommand{\posInts}{\ints_+}
\newcommand{\nats}{\ensuremath{\mathbb{N}}}

\newcommand{\distas}{\sim}
\newcommand{\distiid}{\stk{\distas}{iid}}
\newcommand{\distind}{\stk{\distas}{ind}}

\newcommand{\E}{\mathbb{E}}	% expectation
\renewcommand{\Pr}{\mathbb{P}}	% expectation
\newcommand{\ind}{\mathds{1}} % indicator function
\newcommand{\sgn}{\operatorname{sgn}} % sign function


% \declaretheorem[name=Theorem,numberwithin=section]{theorem}
% \declaretheorem[name=Lemma,numberlike=theorem]{lemma}
% \declaretheorem[name=Proposition,numberlike=theorem]{proposition}
% \declaretheorem[name=Corollary,numberlike=theorem]{corollary}
% \declaretheorem[name=Conjecture,numberlike=theorem]{conjecture}
% \declaretheorem[name=Definition,numberlike=theorem]{definition}
% \declaretheorem[name=Assumption,numberlike=theorem]{assumption}

% shorthand for automatic parenthesis
\newcommand{\lrp}[1]{\left(#1\right)}
\newcommand{\lrb}[1]{\left[#1\right]}
\newcommand{\lrc}[1]{\left\{#1\right\}}
\newcommand{\lra}[1]{\left\langle #1\right\rangle}
% \newcommand{\norm}[1]{\left\lVert #1\right\rVert} % already included in shortex
\newcommand{\abs}[1]{\left\lvert #1\right\rvert} % already included in shortex

% variants
\newcommand{\veps}{\varepsilon}
\newcommand{\vtheta}{\vartheta}
\newcommand{\vpi}{\varpi}
\newcommand{\vrho}{\varrho}
\newcommand{\vsigma}{\varsigma}
\newcommand{\vphi}{\varphi}
\newcommand{\vkappa}{\varkappa}

% parametric distributions
\newcommand{\Norm}{\mathcal{N}}
\newcommand{\StudentT}{\mathcal{T}}
\newcommand{\Wish}{\mathcal{W}}
\newcommand{\InvWish}{\mathcal{IW}}
\newcommand{\Lap}{\distNamed{Lap}}
\newcommand{\ChiSq}{\chi^2}
\newcommand{\Unif}{\distNamed{Unif}}
\newcommand{\Gam}{\distNamed{Gam}}
\newcommand{\Gum}{\distNamed{Gumbel}}
\newcommand{\GEV}{\distNamed{GEV}}
\newcommand{\Cat}{\distNamed{Categorical}}
\newcommand{\InvGam}{\distNamed{InvGam}}
\newcommand{\Poiss}{\distNamed{Poiss}}
\newcommand{\NegBin}{\distNamed{NegBin}}
\newcommand{\Exp}{\distNamed{Exp}}
\newcommand{\Beta}{\distNamed{Beta}}
\newcommand{\Dir}{\distNamed{Dir}}
\newcommand{\Binom}{\distNamed{Binom}}
\newcommand{\Multi}{\distNamed{Multi}}
\newcommand{\Bern}{\distNamed{Bern}}
\newcommand{\Geom}{\distNamed{Geom}}
\newcommand{\Cauchy}{\distNamed{Cauchy}}
\newcommand{\VMF}{\distNamed{vMF}}

% Convenient math shorthand
\newcommand{\doverline}[1]{\overline{\overline{#1}}}
\newcommand{\indicator}[1]{\mathbb{I}_{\{#1\}}}
\newcommand{\identity}{\mathbf{I}}
\newcommand{\transpose}{^{\mathrm{T}}}
\newcommand{\dagg}{^{\dagger}}
\newcommand{\trace}{\mathrm{Tr}}
\newcommand\vm[1]{\bm{\mathrm{#1}}} % vector or matrix
\newcommand{\suchthat}{\mathrm{s.t.}}
\newcommand{\inv}{^{-1}}
\newcommand{\one}{\vm{1}}
\newcommand{\zero}{\vm{0}}
\newcommand{\unit}{\vm{e}}
\newcommand{\mop}[1]{\mathop{\mathrm{#1}}}
\newcommand*\blank{{\mkern 2mu\cdot\mkern 2mu}}

% unary/functions
\let\oldP\P
\renewcommand\P{\relax\ifmmode\mathbb{P}\else\oldP\fi} %make \P be mathbbP in math mode, usual defn as pilcrow in text mode
% \newcommand{\E}{\mathbb{E}}	% expectation
\newcommand{\Var}{\operatorname{Var}}	% variance
\newcommand{\Cov}{\operatorname{Cov}}	% covariance
\newcommand{\Corr}{\operatorname{Corr}}	% correlation
\newcommand{\supp}{\operatorname{supp}} %support

% convergence
\newcommand{\convas}{\overset{a.s.}{\to}}
\newcommand{\convp}{\overset{p}{\to}}
\newcommand{\convd}{\overset{d}{\to}}
\newcommand{\eqd}{\overset{d}{=}}
\newcommand{\eqas}{\overset{a.s.}{=}}

\DeclareMathOperator{\pr}{pr}

%%%%%%%%%%%%%%%%%%%%%%%%%%%%%%%%%%%%%%%%%%%%%%
%%% Calculus 
%%%%%%%%%%%%%%%%%%%%%%%%%%%%%%%%%%%%%%%%%%%%%%
\let\oldd\d
\renewcommand\d{\relax\ifmmode\mathrm{d}\else\oldd\fi} %make \d be mathrm{d} in math mode, usual defn as underdot(?) in text mode
\newcommand{\pd}{\partial}
\newcommand{\grad}{\nabla}
\newcommand{\der}[3][]{\ensuremath{\frac{{\d}^{#1} #2}{\d {#3}^{#1}}}}
\newcommand{\pder}[3][]{\ensuremath{\frac{{\pd}^{#1} #2}{\pd {#3}^{#1}}}}
\newcommand{\hes}[3]{\ensuremath{\frac{\pd^2 #1}{\pd #2 \pd #3}}}
